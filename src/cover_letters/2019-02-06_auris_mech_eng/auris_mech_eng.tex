\documentclass[12pt]{cover_letter}

\date{February 6, 2019}

\usepackage{pdfpages}

\begin{document}
  \begin{letter}{}

    \opening{Dear Auris Health hiring manager,}

    % Sets the current page style to fancy (must happen after "opening")
    \thispagestyle{fancy}

    Your \textit{I \& A Mechanical Engineering Internship - Advanced Development} posting found on the Auris Careers website interests me. A background on various robotic, engineering teams has given me extensive experience in advanced CAD, design for manufacture, and rapid prototyping, which would allow me to make meaningful contributions on your Advanced Development Engineering team.

    I was first exposed to rapid prototyping on my high school FIRST robotics team, of which I served as CAD, engineering, and eventually team captain. In the confines of a short, six-week build period, I improved my ability to rapidly ideate and design for ease of manufacture. In particular, I became particularly adept in both designing for and manufacturing with our in-house manual and CNC equipment. By my senior year, I managed our prototyping team, utilizing CNC routed wood to go from problem, to idea, to manufacturing, to functioning concept in under twenty-four hours. I think I could be an asset to your team in rapidly creating functional prototypes.

    Most of all, I enjoy hands on testing and problem solving. At Georgia Tech, I used laser cutters and 3D printers to build three, iterative revisions of a robot that competed in the Georgia Tech "Creative Design Decisions Competition", in which most teams only were able to build one. In order to ensure that our robot would win a race to a game piece, the team used slow-motion video to optimize our mechanism. The final mechanism revision used a tuned-mass damper to absorb collision energy and geometry that minimized the necessary precision. In the past, I also lead a team of six students in designing and manufacturing a three pound battlebot. The team extensively used CNC machining to create a novel weapon drum that was more reliable and powerful than the equivalent drum manufactured with conventional methods. I believe I would thrive on your collaborative, multi-disciplinary team where I would be challenged on a daily basis.

    Attached you will find my resume; it contains further descriptions of my work as part of various engineering teams at Georgia Tech and in high school. I would be happy to answer further questions by email at \href{mailto:michael.bick@gatech.edu}{\nolinkurl{michael.bick@gatech.edu}}.

    \closing{Sincerely,}

  \end{letter}
\end{document}
