\RequirePackage[T1]{fontenc}
\RequirePackage{lmodern}

% Reformat margins
\RequirePackage[letterpaper, portrait, margin=1in, top=1.5in,
  bottom=0.5in, headheight=0.9in, headsep=0.1in]{geometry}

% Remove numbering
\pagenumbering{gobble}

% Header formatting
\RequirePackage{fancyhdr}
\pagestyle{fancy}

% Load tabularx for auto-width columns
\RequirePackage{tabularx}
% Create a tabularx column type that avoids hyphenation
% https://tex.stackexchange.com/questions/341592/hyphenating-text-inside-tabularx
% \newcolumntype{Z}{>{\raggedright\arraybackslash}X}

\fancyhead{} % clear all fields
\fancyhead[L]{
  \fontsize{10}{12} \selectfont
  \begin{tabularx}{\textwidth}{Xlr}
    \fontsize{20.74}{25}\selectfont
    \textsc{Michael A. Bick}
    % \huge{\textsc{Michael A. Bick}}
    \fontsize{10}{12}\selectfont
    & \begin{tabular}{l}350 Ferst Drive\\
      329889 Georgia Tech Station\\
      Atlanta, GA 30332-1295\\
      US Citizen\end{tabular}
    & \begin{tabular}{r}(747)-227-6723\\
      michael.bick@gatech.edu\end{tabular}
  \end{tabularx}
}
\renewcommand{\headrulewidth}{0.4pt}

% Setup microtype package
\RequirePackage[activate={true,nocompatibility},final,tracking=true,kerning=true,spacing=true,factor=1100,stretch=10,shrink=10]{microtype}
% activate={true,nocompatibility} - activate protrusion and expansion
% final - enable microtype; use "draft" to disable
% tracking=true, kerning=true, spacing=true - activate these techniques
% factor=1100 - add 10% to the protrusion amount (default is 1000)
% stretch=10, shrink=10 - reduce stretchability/shrinkability (default is 20/20)
\SetTracking{encoding={*}, shape=sc}{0}
